% !TeX root = ../dokumentation.tex

\addchap{\langanhang}
In diesem Teil befindet sich noch der Code des Programmes. Dieser befindet sich auch in den zugehörigen Dateien in dem Ordner \textit{Code}.
\section*{Main.cpp}
\begin{lstlisting}[caption=Code aus Main.cpp, label=Bsp.1]
#include <stdio.h>
#include <iostream>
#include <string>
#include <fstream>
#include <vector>
#include <algorithm>
#include <cstdlib>
#include <set>
#include <map>

#include "settings.h"
#include "dempster.h"

using namespace std;

double toDouble(string s) {
	replace(s.begin(), s.end(), ',', '.');
	return atof(s.c_str());
}

void printFile(string filename) {
	ifstream file(filename);
	string value;

	while(file.good()) {
		getline(file, value, ';'); // read a string until next comma: http://www.cplusplus.com/reference/string/getline/
		//cout << string(value, 1, value.length() - 2); // display value removing the first and the last character from it
		cout << string(value) << "-";
	}

	cout << endl;
}

vector<string> readFileInVector(string filename) {
	vector<string> read;
	ifstream file(filename);
	string value;
	int counter = 0; // Every 4th element has a '\n' afterwards. 
						// This needs to be removed from the element in the vector.

	while(file.good()) {
		counter++;

		if((counter % 4) == 0) {
			getline(file, value, CSV_ENDL); // read a string until next comma
		}
		else {
			getline(file, value, CSV_SEP); // read a string until next comma
		}

		read.push_back(value);
	}

	if(read.empty()) {
		cerr << " Error: Vector empty. Probably unreadable file";
	}

	read.pop_back(); //empty element at the end gave me headache

	file.close();

	return read;
}

void writeResultsToFile(string input, string filename, vector<vector<Evidence>> evidences, vector<map<string, double>> plausibilities) {
	if(evidences.size() != plausibilities.size()) {
		cout << " Error: Something went wrong while calculating evidences and plausibilities!" << endl;

		return;
	}
	
	ofstream file(filename);

	file << HEAD_CLK << CSV_SEP
		<< FEAR << CSV_SEP
		<< SURPRISE << CSV_SEP
		<< ANGER << CSV_SEP
		<< JOY << CSV_SEP
		<< DISGUST << CSV_SEP
		<< SORROW << CSV_SEP
		<< HEAD_MAX << CSV_SEP
		<< HEAD_MAXVAL << CSV_ENDL;

	double crt;
	double max;
	string e;

	for(size_t i = 0; i < evidences.size() && file.good(); i++) {
		crt = 0.0;
		max = 0.0;

		file << (i + 1) << CSV_SEP;

		for(const auto& emotion : OMEGA) {
			file << (crt = plausibilities[i][emotion]) << CSV_SEP;

			if(crt > max) {
				e = emotion;
				max = crt;
			}
		}

		file << e << CSV_SEP << max << CSV_ENDL;
	}

	file.close();
	std::cout << "Saved results of data " << input << " in file: " << filename << "." << endl;
}

void printAverageSpeed(vector<string> data) {
	int elem = 0;
	int amountOfElements = 0;
	double average = 0.0, min = 0.0, max = 0.0, temp = 0.0;
	bool init = false;

	for(int i = 4; i < data.size(); i += 4) { //empty 
		elem = i + 1; // Get specific element of Sprechgeschwindigkeit 
		temp = toDouble(data[elem]); // specific element of Sprechgeschwindigkeit
		average += temp;
		amountOfElements++;

		//min max calculation
		if(init == false) { //this just once
			init = true;
			min = temp;
			max = min;
		}

		if(min > temp) {
			min = temp;
		}
		else if(max < temp) {
			max = temp;
		}
	}

	average = average / amountOfElements;

	cout << "The average Speed is: " << average << endl;
	cout << "The max value is: " << max << endl;
	cout << "The min value is: " << min << endl;
}

set<string> getEmotionOfSpeed(double value) {
	set<string> temp;

	if(value <= SLOW) {
		temp.insert(DISGUST); // Ekel
		temp.insert(JOY); // Freude
	}
	else if(value >= FAST) {
		temp.insert(FEAR); // Angst
		temp.insert(SURPRISE); // Ueberraschung
		temp.insert(ANGER); // Wut
		temp.insert(JOY); // Freude
	}
	else { //Speed is normal - so it is between slow and fast (2.5 < x < 6.6)
		temp.insert(ANGER); // Wut
		temp.insert(JOY); // Freude
	}

	return temp;
}

set<string> getEmotionOfPitch(string value) {
	set<string> temp;

	//string.compare returns 0 if equal
	if((value.compare(LOW1) == 0) || (value.compare(LOW2) == 0)) {
		temp.insert(DISGUST); // Ekel
		temp.insert(SORROW); // Traurigkeit
	}
	else if(value.compare(MEDIUM) == 0) {
		temp.insert(JOY); // Freude
	}
	else if((value.compare(HIGH1) == 0) || (value.compare(HIGH2) == 0)) {
		temp.insert(FEAR); // Angst
		temp.insert(SURPRISE); // Ueberraschung
		temp.insert(ANGER); // Wut
		temp.insert(JOY); // Freude
	}
	else {
		cerr << " Error: Could not read the data. Tried to read the following: "
			<< value << " . Are you sure that you set all macros right?" << endl;
	}

	return temp;
}

set<string> getEmotionOfIntensity(string value) {
	set<string> temp;

	//string.compare returns 0 if equal
	if((value.compare(LOW1) == 0) || (value.compare(LOW2) == 0)) {
		temp.insert(DISGUST); // Ekel
		temp.insert(SORROW); // Traurigkeit
	}
	else if(value.compare(MEDIUM) == 0) {
		temp.insert(JOY); // Freude
		temp.insert(DISGUST); // Ekel
	}
	else if((value.compare(HIGH1) == 0) || (value.compare(HIGH2) == 0)) {
		temp.insert(SURPRISE); // Ueberraschung
		temp.insert(ANGER); // Wut
		temp.insert(JOY); // Freude
	}
	else {
		cerr << " Error: Could not read the data. Tried to read the following: "
			<< value << " . Are you sure that you set all macros right?" << endl;
	}

	return temp;
}

map<string, double> plausibility(vector<Evidence> data) {
	std::cout << "Plausibility:" << endl;

	map<string, double> plausibility;

	//init
	for(const auto& emotion : OMEGA) {
		plausibility[emotion] = 0.0;
	}

	plausibility[PL_MAX] = 0.0;

	for(size_t i = 0; i < data.size(); i++) {
		//for each emotion
		for(const auto& emotion : OMEGA) {
			if(data[i].emotions.find(emotion) != data[i].emotions.end()) {
				//found element
				plausibility[emotion] += data[i].value;
			}
		}
	}

	//print
	string e;
	bool init = false;
	for(const auto& emotion : OMEGA) {
		if(init == false) {
			init = true;
			plausibility[PL_MAX] = plausibility[emotion];
			e = emotion;
		}

		if(plausibility[PL_MAX] < plausibility[emotion]) {
			plausibility[PL_MAX] = plausibility[emotion];
			e = emotion;
		}

		std::cout << "PI for Emotion: " << emotion << " : " << plausibility[emotion] << " " << endl;
	}

	std::cout << "Max PI found in Emotion: " << e << " : " << plausibility["max"] << endl;

	return plausibility;
}

void calculate_plausibilities(string input, string output) {
	vector<string> data = readFileInVector(input);

	vector<vector<Evidence>> evidences;
	vector<map<string, double>> plausibilities;

	for(size_t i = DATA_START; i < data.size(); i = i + 4) {
		cout << "Takt " << data[i] << endl;

		set<string> speedEmotions = getEmotionOfSpeed(toDouble(data[i + 1]));
		double speedConfidence = CONF_SPEED;

		set<string> pitchEmotions = getEmotionOfPitch(data[i + 2]);
		double pitchConfidence = CONF_PITCH;

		set<string> intensityEmotions = getEmotionOfIntensity(data[i + 3]);
		double intensityConfidence = CONF_INTENSITY;

		evidences.push_back(
			dempster(speedEmotions, speedConfidence,
						pitchEmotions, pitchConfidence,
						intensityEmotions, intensityConfidence));

		plausibilities.push_back(plausibility(evidences.back()));

		std::cout << endl;
	}

	writeResultsToFile(input, output, evidences, plausibilities);

	cout << " Note: Done." << endl;
}

int main() {
	global_init();

	char pause;
	std::cout << "Press 'enter' to start with the file " << FILE1 << "." << endl;
	pause = getchar();
	calculate_plausibilities(FILE1, OUTPUT1);
		
	std::cout << "Press 'enter' to continue with the file " << FILE2 << "." << endl;
	pause = getchar();
	calculate_plausibilities(FILE2, OUTPUT2);

	std::cout << "To exit this programm press 'enter'." << endl;
	pause = getchar();

	return 0;
}
\end{lstlisting}

\section*{dempster.h}
\begin{lstlisting}[caption=Code dempster.h]
#ifndef _DEMPSTER_H
#define _DEMPSTER_H

#include <vector>
#include <string>
#include <set>
#include "settings.h"

namespace std {

	struct Evidence {
		Evidence() {} // to dynamically add elements to a vector 
		set<string> emotions;
		double value = 0.0;
	};

	vector<Evidence> dempster(set<string> speedEmotions, double speedConfidence,
							  set<string> pitchEmotions, double pitchConfidence,
							  set<string> intensityEmotions, double intensityConfidence) {

		// Initialization of the three evidences
		vector<Evidence*> m_1;
		Evidence m_1a;
		Evidence m_1o; // Omega
		vector<Evidence*> m_2;
		Evidence m_2a;
		Evidence m_2o; // Omega
		vector<Evidence*> m_3;
		Evidence m_3a;
		Evidence m_3o; // Omega

					   // m_1: Speed
		m_1a.emotions = speedEmotions;
		m_1a.value = speedConfidence;
		m_1o.emotions = OMEGA;
		m_1o.value = (1 - m_1a.value);
		m_1.push_back(&m_1a);
		m_1.push_back(&m_1o); //OMEGA - index 1

							  // m_2: Pitch
		m_2a.emotions = pitchEmotions;
		m_2a.value = pitchConfidence;
		m_2o.emotions = OMEGA;
		m_2o.value = (1 - m_2a.value);
		m_2.push_back(&m_2a);
		m_2.push_back(&m_2o); //OMEGA - index 1

							  // m_3: Intensity
		m_3a.emotions = intensityEmotions;
		m_3a.value = intensityConfidence;
		m_3o.emotions = OMEGA;
		m_3o.value = (1 - m_3a.value);
		m_3.push_back(&m_3a);
		m_3.push_back(&m_3o); //OMEGA - index 1

							  // m_1 U m_2 = m_12
		vector<Evidence> m_12; // Call by copy required - variables are set by lower scope (for loop)

		for(size_t i = 0; i < m_1.size(); i++) {
			for(size_t j = 0; j < m_2.size(); j++) {
				Evidence temp;
				temp.value = m_1[i]->value * m_2[j]->value;

				set_intersection(m_1[i]->emotions.begin(), m_1[i]->emotions.end(),
								 m_2[j]->emotions.begin(), m_2[j]->emotions.end(),
								 inserter(temp.emotions, temp.emotions.begin()));

				m_12.push_back(temp);
			}
		}

		// Check for empty set and correct if necessary
		// (only possible once: in intersection m_1 U m_2 without Omega)
		for(size_t i = 0; i < m_12.size(); i++) {
			if(m_12[i].emotions.empty()) {
				cout << " Note: Found empty set in m_12" << endl;

				double k = 1 / (1 - m_12[i].value); // calculate correction factor k

				m_12.erase(m_12.begin() + i); // delete empty element i

				for(size_t a = 0; a < m_12.size(); a++) { // correct remaining elements with k
					m_12[a].value *= k;
				}

				cout << " Note: Corrected m_12 with value k=" << k << endl;

				break; // leave for loop since there can only be one correction
			}
		}

		// m_12 U m_3 = m_123
		vector<Evidence> m_123; //call by copy required here - because the variables are created in a loop dynamically - do not save only pointers in this vector

		for(size_t i = 0; i < m_12.size(); i++) {
			for(size_t j = 0; j < m_3.size(); j++) {
				Evidence temp; //empty element
				temp.value = (m_12[i].value) * (m_3[j]->value);

				set_intersection(m_12[i].emotions.begin(), m_12[i].emotions.end(),
								 m_3[j]->emotions.begin(), m_3[j]->emotions.end(),
								 inserter(temp.emotions, temp.emotions.begin()));

				m_123.push_back(temp);
			}
		}

		// Check for empty set and correct if necessary
		// (multiple empty sets possible)
		bool correction = false;
		double sum_empty = 0.0;
		vector<size_t> index_to_delete;

		for(size_t k = 0; k < m_123.size(); k++) {
			if(m_123[k].emotions.empty()) {
				cout << " Note: Found empty set in m_123" << endl;

				correction = true;
				sum_empty += m_123[k].value;

				index_to_delete.push_back(k); //remember which element needs to be removed
			}
		}

		// correct if empty set found
		if(correction) {
			double k = 1 / (1 - sum_empty);

			// delete empty elements
			for(size_t b = 0; b < index_to_delete.size(); b++) {
				m_123.erase(m_123.begin() + index_to_delete[b]);
			}

			// correct remaining elements with k
			for(size_t a = 0; a < m_123.size(); a++) {
				m_123[a].value *= k;
			}

			cout << " Note: Corrected m_123 with k=" << k << endl;
		}

		return m_123;
	}

}

#endif // _DEMPSTER_H
\end{lstlisting}


\section*{settings.h}
\begin{lstlisting}[caption=Code settings.h]
/*
	This file contains all settings
*/

#ifndef _SETTINGS_H
#define _SETTINGS_H

// I/O File Settings
#define FILE1 "E_004.csv" // Path of 1st file
#define FILE2 "E_004b.csv" // Path of 2nd file
//#define FILE3 "E_004c.csv" // Path of 3rd file - Not applieable for this project

#define OUTPUT1 "resulta.csv" // Path of Output File
#define OUTPUT2 "resultb.csv" //Path of second output file

#define DATA_START 4 //the data starts at element of 4, or index 4 (to skip header)

// Definitions of Emotions
#define FEAR "A" // Angst
#define SURPRISE "U" // Ueberraschung
#define ANGER "W" // Wut
#define JOY "F" // Freude 
#define DISGUST "E" // Ekel
#define SORROW "T" // Traurigkeit

// Definitions of Confidences
#define CONF_SPEED 0.6
#define CONF_PITCH 0.8
#define CONF_INTENSITY 0.7

// Definitions 
// ...for Speed 
#define SLOW 2.5 //untere Grenze
#define FAST 5.6 //obere Grenze

// ...for Pitch and Intensity
#define LOW1 "sehr niedrig"
#define LOW2 "niedrig"
#define MEDIUM "normal"
#define HIGH1 "hoch"
#define HIGH2 "sehr hoch"

// Definitions for Plausibility
#define PL_MAX "max"

// Definitions for Output File
#define CSV_SEP ';'
#define CSV_ENDL '\n'
#define HEAD_CLK "Takt"
#define HEAD_MAX "Max"
#define HEAD_MAXVAL "PL of Max"

namespace std {

	set<string> OMEGA;

	void global_init() {
		OMEGA.insert(FEAR);
		OMEGA.insert(SURPRISE);
		OMEGA.insert(ANGER);
		OMEGA.insert(JOY);
		OMEGA.insert(DISGUST);
		OMEGA.insert(SORROW);
	};

}

#endif // _SETTINGS_H
\end{lstlisting}