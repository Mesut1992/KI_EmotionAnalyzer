\chapter{Einleitung}
Die Vorraussetzung für die Anwendung der Evidenztheorie ist gegegeben, da folgendendes gilt:

\begin{enumerate}
  \item Jegliche Alternativmengen sind gegeben. Das bedeutet, dass davon ausgegangen wird, dass einer der Grundemotionen stets auf den Benutzer zu trifft. 
  \item Alternativen schließen sich gegenseitig aus: Falls eine Emotion \textit{erkannt} wurde, schließt diese, andere Emotionen aus.
\end{enumerate}


\textbf{SPÄTER EVENTUELL ANPASSEN:}
Der Aufbau dieser Arbeit ist umfasst zuerst das Design des Programmes. Im Design werden die Emotionsklassen konkretisiert, fehlende Werte für Konfidenz hinzugefügt und letztendlich wird die Implementation betrachtet.  