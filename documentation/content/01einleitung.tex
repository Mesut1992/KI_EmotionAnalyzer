\chapter{Einleitung}
Die Voraussetzungen für die Anwendung der Evidenztheorie ist gegegeben, da folgendes gilt:

\begin{enumerate}
  \item Jegliche Alternativmengen sind gegeben. Das bedeutet, dass davon ausgegangen wird, dass eine der Grundemotionen stets auf den Benutzer zu trifft und alle Grundemotionen vom Programm erkannt werden können. 
  \item Jegliche Alternativen schließen sich gegenseitig aus: Falls eine Grundemotion \textit{erkannt} wurde, schließt diese andere aus; d.h. eine Tonaufnahme eines Benutzers beinhaltet immer nur eine der Grundemotionen.
\end{enumerate}

Der Aufbau dieser Arbeit umfasst vorerst das Programmdesign. Hierbei werden die Ausprägungen der Emotionen konkretisiert, fehlende Werte für Konfidenzen hinzugefügt und mögliche Emotionsklassen mit den zugehörigen Merkmalen tabellarisiert. Daraufhin folgt die Dokumentation der Implementation und eine Beschreibung.