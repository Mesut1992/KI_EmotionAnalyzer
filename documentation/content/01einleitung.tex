\chapter{Einleitung}
Die Vorraussetzungen für die Anwendung der Evidenztheorie ist gegegeben, da folgendendes gilt:

\begin{enumerate}
  \item Jegliche Alternativmengen sind gegeben. Das bedeutet, dass davon ausgegangen wird, dass einer der Grundemotionen stets auf den Benutzer zu trifft und alle Grundemotionen vom Programm erkannt werden können. 
  \item Jegliche Alternativen schließen sich gegenseitig aus: Falls eine Emotion \textit{erkannt} wurde, schließt diese andere Grundemotionen aus.
\end{enumerate}


Der Aufbau dieser Arbeit umfasst vorerst das Design des Programmes. Im Design werden die Ausprägungen der Emotionen konkretisiert, fehlende Werte für Konfidenz hinzugefügt und mögliche Emotionsklassen mit den zugehörigen Merkmalen tabellarisiert. Daraufhin folgt die Dokumentation der Implementation. 