\chapter{Starten der Applikation}
Um diese Applikation zu starten können sind zwei Wege möglich. 
Der erste Weg ist der einfache, jedoch beruht dieser auf (sehr) viel Vertrauen. Im Submission-Ordner lässt sich der Ordner \textit{Build} finden, in welchem das Programm schon ausführbar als .EXE Datei bereit steht. Diese Datei sollte auf Windows Betriebssystem ausführbar sein.  

Der zweite Weg ist etwas umständlicher, jedoch ist weniger Vertrauen gegenüber dem Entwicklerteam notwendig. Im Submission-Ordner lassen sich im Ordner \textit{Code} die vorgestellten Codedateien finden. Bei der Kompilierung der Datei \textit{Main.cpp}, ein C++ Programm, muss darauf geachtet werden, dass \textit{settings.h} und \textit{dempster.h} verlinkt werden.


Wichtig bei der Ausführung des Programmes ist, dass die Dateien \verb|E_004.csv| und \verb|E_004b.csv| im selben Order sind. Ansonsten hat das Programm keine Daten zu verarbeiten.  