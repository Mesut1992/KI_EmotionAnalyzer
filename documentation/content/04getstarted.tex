\chapter{Starten der Applikation}
Um diese Applikation starten können sind zwei Wege möglich. 
Der erste Weg ist der einfachere, jedoch beruht dieser auf (sehr) viel Vertrauen gegenüber dem Entwicklerteam: Im Submission-Ordner befindet sich der Ordner \textit{Build}, in dem das Programm schon ausführbar als .EXE Datei bereit gestellt ist. Diese Datei sollte auf Windows Betriebssystem blind ausführbar sein. Das bedeutet, dass der Benutzer lediglich die .EXE Datei ausführen könnte.

Der zweite Weg ist etwas umständlicher, jedoch ist weniger Vertrauen gegenüber dem Entwicklerteam notwendig. Im Submission-Ordner befindet sich zusätzlich der Ordner \textit{Code}, welcher die vorgestellten Codedateien beinhaltet. Diese müssen vom Benutzer selbst Kompiliert werden. Bei der Kompilierung der Datei \textit{Main.cpp} muss darauf geachtet werden, dass \textit{settings.h} und \textit{dempster.h} verlinkt werden.


Wichtig bei der Ausführung des Programmes ist, dass die Dateien \verb|E_004.csv| und \verb|E_004b.csv| im selben Order sind, da ansonsten keine Daten zum Verarbeiten vorliegen.  