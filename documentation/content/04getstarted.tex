\chapter{Starten der Applikation}
Bei der Ausführung der Applikation ist zu beachten, dass die Eingabedateien sich mit dem korrekten Namen im korrekten Ordner befinden. Laut den Standardeinstellungen in der \verb|settings.h|-Datei sollten sich die beiden Dateien im selben Ordner wie das Programm befinden und \verb|E_004.csv| bzw. \verb|E_004b.csv| genannt sein. Außerdem muss das Programm Schreibrechte auf die Ausgabedatei haben; laut Standardeinstellungen sind dies \verb|resulta.csv| und \verb|resultb.csv| im selben Ordner wie das Programm.

Es gibt zwei Möglichkeiten, um das Programm zu starten. Zum Einen kann die schon kompilierte .EXE-Datei auf einem Windows-PC ausgeführt werden, und zum Anderen kann auf Basis des Quellcodes das Programm noch einmal neu kompiliert werden. Wenn \verb|Main.cpp| kompiliert wird, ist darauf zu achten, dass \verb|settings.h| und \verb|dempster.h| verlinkt werden. Alternativ zu einer Kompilierung "von Hand" kann mit Microsoft Visual Studio die Solution (befindlich im Unterordner "Solution") geöffnet werden.
