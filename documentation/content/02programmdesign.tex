\chapter{Abbildung der Evidenzen}
Dieses Kapitel befasst sich mit der Zuordnung jeder Emotion und dessen Attribute. Vorerst wird diese Zuordnung der Aufgabenstellung gemäß dargestellt. Daraufhin werden die Werte der Beispieldaten näher betrachtet und die zuvor vorgestellte Zuordnung der Emotionen konkretisiert. 

\section{Emotionen}
Die Tabelle \ref{tab:emotionenAbbildung} stellt die Einstufung der Emotionen aus der Aufgabenstellung dar. Dabei fällt auf, dass die Schallstärke nicht ausschlaggebend für die Emotion \textit{Angst} ist und dass die Sprechgeschwindigkeit nicht auf Traurigkeit zutreffend ist. Außerdem fällt auf, dass keine Mittelstufe in den Ausprägungen der Merkmale existiert. Diese Tabelle wird im weiteren Verlauf konkretisiert und an die vorgegebenen Beispieldaten angepasst.

\begin{table}[h]
\begin{tabular}{ l | c | c | c | c}
  Emotion & Kürzel & Sprechgeschwindigkeit & Tonlage & Schallstärke \\
  \hline 
  Angst & A & Schnell & Hoch & - \\
  Überraschung & U & Schnell & Hoch & Hoch \\
  Wut & W & Schnell & Hoch & Hoch \\
  Freude & F & Schnell / Langsam & Hoch & Hoch \\
  Ekel & E & Langsam & Tief & Schwach \\
  Traurigkeit & T & - & Tief & Schwach \\
\end{tabular}
\caption{Abbildung der Emotionen nach Aufgabenstellung}
\label{tab:emotionenAbbildung}
\end{table}

\section{Nutzung der Beispieldaten}
Das Projekt benutzt die Beispielwerte aus den Dateien \verb|E_004.csv| und \verb|E_004b.csv|. In beiden Dateien sind jeweils 50 Takte angegeben, die zu analysieren sind. 
%Das bedeutet, dass das Programm bei Durchführung, 50 Emotionen als Ergebnis ausgeben sollte.


Es gibt dabei jeweils immer drei Evidenzen, die in jedem Takt, entstehen. In der Aufgabenstellung geht zwar hervor, dass jeweils immer nur 3 Levels von Ausprägungen der jeweiligen Evidenz erkannt werden muss (so wie oben in der Tabelle gekennzeichnet), jedoch sind in den Beispieldaten in der Tonlage und in der Schallstärke weitere Ausprägungen zu finden. Die folgende Liste enthält alle Ausprägungen der Merkmale, die in den Beispieldaten zu finden sind. 

\begin{description}
  \item [Sprechgeschwindigkeit in Silben/Sekunden] langsam, normal, schnell
  \item [Durchnschnittliche Tonlage] sehr niedrig, niedrig, normal, hoch, sehr hoch
  \item [Schallstärke] sehr niedrig, niedrig, normal, hoch, sehr hoch
\end{description}

Da die Anzahl der Ausprägungen in den Beispieldaten größer ist, als in der Aufgabenstellung, befasst sich der verbleibende Teil dieses Kapitels mit der Nutzung und Einteilung der Beispieldaten.

\section{Normierung der Ausprägungen der Merkmale}
Jegliche Normierungen sind über eine externe Settingsdatei statisch festgelegt. Diese können, unabhängig vom Programmcode, vor Programmstart angepasst werden. Der Programmaufbau wird im nächsten Kapitel näher betrachtet.

\subsection{Sprechgeschwindigkeit}
\label{sprechgeschwindigkeit_auspr}
Nach einer Analyse der Werte der Sprechgeschwindigkeit ist festgestellt worden, dass die Werte in beiden Dateiein von 2 bis 7 variieren. Die folgende Tabelle \ref{tab:sprechgeschwdingikeit} stellt die Analyse dar. 

\begin{table}[h]
\begin{tabular}{ c | c | c | c}
  Dateiname & Minimum & Maximum & Durchschnitt \\
  \hline 
  \verb|E_004.csv|  & 2.6 & 5.9 & 4 \\
  \verb|E_004b.csv| & 2.9 & 6.5 & 4.442 \\
\end{tabular}
\caption{Ausprägung der Sprechgeschwindigkeit}
\label{tab:sprechgeschwdingikeit}
\end{table}

Dadurch wurde folgende Norm für die Ausprägung der Sprechgeschwindigkeit aufgestellt, welche für beide Beispieldaten gilt:
\begin{description}
  \item [langsam] Der Wert ist kleiner als 2.5
  \item [normal] Der Wert liegt im Intervall: 2.6 - 5.5
  \item [schnell] Der Wert ist größer als 5.6
\end{description}

Die ausgewählten Werte orientieren sich von den durchschnittswerten aus der Analyse der Beispieldatei. Die Analyse findet sich auch im Programmcode um spätere Analysen für andere Daten zu gewährleisten.

\subsection{Durchschnittliche Tonlage und Schallstärke}
Die Ausprägung für die Schallstärke und die durchschnittliche Tonlage wird wie folgt zusammengefasst:
\begin{description}
\item [niedrig] falls sehr niedrig oder niedrig
\item [normal] falls normal
\item [hoch] falls hoch oder sehr hoch
\end{description}
Durch diese Normierung werden die Beispieldaten an die Aufgabenstellung angepasst: Es sind nur noch drei Ausprägungen zu analysieren.

\section{Konkretisierung der Emotionen}
Da nun weitere Ausprägungen durch die Beispieldaten dazugekommen sind, ist es notwendig, die Zuordnung der Emotionen zu konkretisieren. Die folgende Tabelle \ref{tab:konkretisierteEmotionen} stellt die finale Zuordnung der Emotionen dar. Unterschiede zu der vorherigen Tabelle \ref{tab:emotionenAbbildung} sind mit fetter Schrift makiert. In der Tabelle ist zu erkennen, dass die Ausprägung \textit{Normal} hinzugefügt worden ist. Diese Ausprägung steht in der Meinung des Entwicklerteams im starken Zusammenhang mit der Emotion \textit{Freude}. 
%Zusätzlich wurde im Merkmal \textit{Schallstärke} die Emotion \textit{Angst} hinzugefügt.


\begin{table}[h]
\begin{tabular}{ l | c | c | c}
  Emotion & Sprechgeschwindigkeit & Tonlage & Schallstärke \\
  \hline 
  Angst A & Schnell & Hoch & - \\
  Überraschung U & Schnell & Hoch & Hoch \\
  Wut W & \textbf{Normal}/Schnell & Hoch & Hoch \\
  Freude F & Langsam/\textbf{Normal}/Schnell & \textbf{Normal}/Hoch &\textbf{Normal}/Hoch\\
  Ekel E & Langsam & Tief & Schwach/\textbf{Normal} \\
  Traurigkeit T & - & Tief & Schwach \\
\end{tabular}
\caption{Konkretisierung der Emotionen}
\label{tab:konkretisierteEmotionen}
\end{table}

\section{Konfidenz der Messungen}
\label{konfidenzfestlegung}
Da keine Zuverlässigkeit der Messung in der Aufgabenstellung angegeben ist, müssen diese vom Entwicklerteam fiktiv geschätzt werden. Diese Werte können statisch vor Programmstart leicht geändert werden. Die folgende Liste stellt die ausgesuchten Werte der Konfidenz der jeweiligen Messungen dar:
\begin{description}
\item [Sprechgeschwindigkeit] liegt bei 0.6
\item [Tonlage] liegt bei 0.8
\item [Schallstärke] liegt bei 0.7
\end{description}


\section{Ergebnisauswertung}
Durch Akkumulation der Evidenzen führt das Programm zu einer Schlussfolgerung. Es wird die Emotion ausgewählt, dessen Plausabilität am höchsten ist. Da beide Dateien jeweils 50 Takte beinhalten, sollten im Idealfall jeweils 50 Emotionen ausgegeben werden, dessen Plausabilität des jeweiligen Taktes, am höchsten ist.



